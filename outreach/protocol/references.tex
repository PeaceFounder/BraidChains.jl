\begin{thebibliography}{9}

\bibitem{ZeroCoin2013} I. Miers, C. Garman, M. Green and A. D. Rubin, "Zerocoin: Anonymous Distributed E-Cash from Bitcoin," 2013 IEEE Symposium on Security and Privacy, Berkeley, CA, 2013, pp. 397-411
\bibitem{ZeroCash2014} Eli Ben-Sasson, Alessandro Chiesa, Christina Garman, Matthew Green, Ian Miers, Eran Tromer, Madars Virza, Zerocash: Decentralized Anonymous Payments from Bitcoin, proceedings of the IEEE Symposium on Security \& Privacy (Oakland) 2014, 459-474, IEEE, 2014
\bibitem{Ryan2014} Mauw S., Radomirović S., Ryan P.Y.A. (2014) Security Protocols for Secret Santa. In: Christianson B., Malcolm J. (eds) Security Protocols XVIII. Security Protocols 2010. Lecture Notes in Computer Science, vol 7061. Springer, Berlin, Heidelberg. 
\bibitem{Adida2016} Ben Adida. 2006. Advances in cryptographic voting systems. Ph.D. Dissertation. Massachusetts Institute of Technology, USA. Advisor(s) Ronald L. Rivest.
\bibitem{Anderson1995} Anderson R., Needham R. (1995) Programming Satan's computer. In: van Leeuwen J. (eds) Computer Science Today. Lecture Notes in Computer Science, vol 1000. Springer, Berlin, Heidelberg
\bibitem{Chaum1981} Chaum, D. L. (1981). Untraceable electronic mail, return addresses, and digital pseudonyms. Communications of the ACM, 24(2), 84-90.
\bibitem{CoinShuffle2014} Ruffing, T., Moreno-Sanchez, P., \& Kate, A. (2014, September). Coinshuffle: Practical decentralized coin mixing for bitcoin. In European Symposium on Research in Computer Security (pp. 345-364). Springer, Cham.
\bibitem{Danezis2003} Serjantov, Andrei \& Danezis, George. (2003). Towards an Information Theoretic Metric for Anonymity. 2482. 10.1007/3-540-36467-6\_4.
\bibitem{Danezis2008} Danezis, G., \& Diaz, C. (2008). A survey of anonymous communication channels (Vol. 27, p. 30). Technical Report MSR-TR-2008-35, Microsoft Research.
\bibitem{DolevYao1983} Dolev, D., \& Yao, A. (1983). On the security of public key protocols. IEEE Transactions on information theory, 29(2), 198-208.
\bibitem{Franck2008} Christian Franck (2008). New Directions for Dining Cryptographers. MSc thesis.
\bibitem{GolleJuels2004} Golle, P., \& Juels, A. (2004, May). Dining cryptographers revisited. In International Conference on the Theory and Applications of Cryptographic Techniques (pp. 456-473). Springer, Berlin, Heidelberg.
\bibitem{Gritzalis2002} Dimitris Gritzalis (2002) Secure Electronic Voting
\bibitem{PublicEvidence2017} Bernhard M. et al. (2017) Public Evidence from Secret Ballots. In: Krimmer R., Volkamer M., Braun Binder N., Kersting N., Pereira O., Schürmann C. (eds) Electronic Voting. E-Vote-ID 2017. Lecture Notes in Computer Science, vol 10615. Springer, Cham 
\bibitem{Rivest2008} Rivest, R. L. (2008). On the notion of ‘software independence’in voting systems. Philosophical Transactions of the Royal Society A: Mathematical, Physical and Engineering Sciences, 366(1881), 3759-3767.
\bibitem{Sampigethaya2006} Sampigethaya, K., \& Poovendran, R. (2006). A survey on mix networks and their secure applications. Proceedings of the IEEE, 94(12), 2142-2181.
\bibitem{USVote2015} Dzieduszycka-Suinat, S., Ph., I.M., Kiniry, J., Zimmerman, D.M., Wagner, D., Robinson, P., \& Adam (2015). The Future of Voting End-to-end Verifiable Internet Voting Specification and Feasibility Assessment Study Internet Voting Today No Guarantees End-to-end Verifiability E2e-viv.
\bibitem{Juvonen2019} Juvonen, Atte. 2019. A framework for comparing the security of voting schemes.
  
\end{thebibliography}
%%% Local Variables:
%%% mode: latex
%%% TeX-master: t
%%% End:
